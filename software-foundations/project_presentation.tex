\documentclass{beamer}
%
% Choose how your presentation looks.
%
% For more themes, color themes and font themes, see:
% http://deic.uab.es/~iblanes/beamer_gallery/index_by_theme.html
%
\mode<presentation>
{
  \usetheme{default}      % or try Darmstadt, Madrid, Warsaw, ...
  \usecolortheme{default} % or try albatross, beaver, crane, ...
  \usefonttheme{default}  % or try serif, structurebold, ...
  \setbeamertemplate{navigation symbols}{}
  \setbeamertemplate{caption}[numbered]
} 

\usepackage[english]{babel}
\usepackage[utf8]{inputenc}
\usepackage[T1]{fontenc}

\title[Your Short Title]{Your Presentation}
\author{You}
\institute{Where You're From}
\date{Date of Presentation}

\begin{document}

\begin{frame}
  \titlepage
\end{frame}

% Uncomment these lines for an automatically generated outline.
%\begin{frame}{Outline}
%  \tableofcontents
%\end{frame}


\begin{frame}{An Opinion on How Language Works}

\begin{itemize}
\item
The verb is head of sentence or clause.
\item
Words combine with other words to form larger units of meaning - compositionality.
\item
All the arguments of a verb combine with it simultaneously.
\end{itemize}

Okay but how do we model this?

\end{frame}


\begin{frame}{According to Montague}

Syntactic algebra, semantic algebra, homomorphism

\end{frame}


\begin{frame}{So Our Syntactico-Semantic Algebra Is...}

\begin{itemize}
\item
Categories? No
\item
Lambda calculus? No
\item
Linear algebra? Wait how is that a syntactic-
\end{itemize}

\end{frame}


\begin{frame}{Pi Calculus, I Choose You!}

The model:

\begin{itemize}
\item
Words are processes that contain some information.
\item
These process have state.
\item
These processes communicate with each other (other words in the sentence) via channels.
\item
These processes can change state on exchanging certain information.
\end{itemize}

\end{frame}

\begin{frame}{Pi Calculus: The Syntax}

Some pseudocode to illustrate the syntax.

\begin{verbatim}
data ArgRole = Nominative | Accusative

data Channel = Output ArgRole | Input ArgRole | Null

data Process = Null 
			 | Prefix Process
			 | Sum Process Process
			 | Parallel Process Process
			 | Scoped Channel Process
			 | Name String [Channel] Process  
\end{verbatim}

\end{frame}


\begin{frame}{CCS: The Syntax}

Some pseudocode to illustrate the syntax.

\begin{verbatim}
data Channel = Offers ArgRole | Accepts ArgRole  
               
data Process = Name 
			 | Comm Prefix Process
			 | Sum Process Process
			 | Parallel Process Process
			 | Scoped Channel Process
			 | Name String [Channel] Process  
\end{verbatim}

Excluding relabelling, identifier and conditional expressions.

\end{frame}


\begin{frame}{CCS: The Semantics}

\begin{verbatim}
type Name = String

data ArgRole = Nominative | Accusative
			 
ram = Comm (Offers Nominative) "ram-subj"
phal = Comm (Offers Accusative) "phal-obj"
khaana = Comm (Accepts Nominative) "khaata-int"
khaana = Comm (Accepts Accusative) "khaata-tr-inc"
khaata-int = Comm (Accepts Accusative) "khaata-tr"
\end{verbatim}

The intended trace of the above system when run is:

\begin{verbatim}

ram | khaana = ram-subj | khaata-int

\end{verbatim}

\end{frame}


\end{document}