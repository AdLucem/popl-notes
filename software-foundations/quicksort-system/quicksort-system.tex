% --------------------------------------------------------------
% This is all preamble stuff that you don't have to worry about.
% Head down to where it says "Start here"
% --------------------------------------------------------------
 
\documentclass[12pt]{article}
 
\usepackage[margin=1in]{geometry} 
\usepackage{amsmath,amsthm,amssymb,scrextend}
\usepackage{fancyhdr}
\pagestyle{fancy}

 
\newcommand{\N}{\mathbb{N}}
\newcommand{\Z}{\mathbb{Z}}
\newcommand{\I}{\mathbb{I}}
\newcommand{\R}{\mathbb{R}}
\newcommand{\Q}{\mathbb{Q}}
\renewcommand{\qed}{\hfill$\blacksquare$}
\let\newproof\proof
\renewenvironment{proof}{\begin{addmargin}[1em]{0em}\begin{newproof}}{\end{newproof}\end{addmargin}\qed}
% \newcommand{\expl}[1]{\text{\hfill[#1]}$}
 
\newenvironment{theorem}[2][Theorem]{\begin{trivlist}
\item[\hskip \labelsep {\bfseries #1}\hskip \labelsep {\bfseries #2.}]}{\end{trivlist}}
\newenvironment{lemma}[2][Lemma]{\begin{trivlist}
\item[\hskip \labelsep {\bfseries #1}\hskip \labelsep {\bfseries #2.}]}{\end{trivlist}}
\newenvironment{problem}[2][Problem]{\begin{trivlist}
\item[\hskip \labelsep {\bfseries #1}\hskip \labelsep {\bfseries #2.}]}{\end{trivlist}}
\newenvironment{exercise}[2][Exercise]{\begin{trivlist}
\item[\hskip \labelsep {\bfseries #1}\hskip \labelsep {\bfseries #2.}]}{\end{trivlist}}
\newenvironment{reflection}[2][Reflection]{\begin{trivlist}
\item[\hskip \labelsep {\bfseries #1}\hskip \labelsep {\bfseries #2.}]}{\end{trivlist}}
\newenvironment{proposition}[2][Proposition]{\begin{trivlist}
\item[\hskip \labelsep {\bfseries #1}\hskip \labelsep {\bfseries #2.}]}{\end{trivlist}}
\newenvironment{corollary}[2][Corollary]{\begin{trivlist}
\item[\hskip \labelsep {\bfseries #1}\hskip \labelsep {\bfseries #2.}]}{\end{trivlist}}
 
\begin{document}
 
% --------------------------------------------------------------
%                         Start here
% --------------------------------------------------------------

\lhead{Quicksort As A Transition System}
\chead{Atreyee Ghosal 20161167}
\rhead{\today}
 
% \maketitle

\section{Data Structures and Notation}

\subsection{List}

The concept of 'List' here is used to refer to a one-dimensional vector. The following operations are referred to:

\begin{itemize}
\item []
$V_1 ++ V_2$ : refers to the concatenation of two vectors $V_1$ of dimensions $1 x m$ and $V_2$ of dimensions $1 x n$, such that the final vector has the dimensions $1 x (m + n)$
\item []
$V [ i : j ]$ : refers to a vector composed of elements at indices $i, i+1....j$ from the original vector
\item []
$length(V)$ : refers to a function that returns the length of the given vector
\end{itemize}

\subsection{Stack}

\section{State Space and Initial State}

\subsection{Initial State}

Initial state is a tuple whose first element is a list of integers $ls$, and second element is a $stack$ with the tuple $(1, length(ls)$ as its only element.

i.e: $X_0 = (ls, [(1, length(ls)])$

\subsection{Final State}

The final state of the transition system is when the stack is empty.

\section{Output Space}

Output space is the same as the input space, necessitating no mapping function between them.

\section{Transition Function}

Define a sub-function \textbf{Partition} as the following, where $++$ denotes the list-append operation:

\begin{itemize}
\item []
$Partition (L, start, end, pivot) :$
	\begin{itemize}
	\item []
	$ls := L[start : end]$
	\item []
	$smaller = \{ x | x \in ls, x < pivot \}$
	\item []
	$greater = \{ x | x \in ls, x \geq pivot \}$
	\item []
	$\forall i, start \leq i \leq end$:
		\begin{itemize}
			\item []
			$L[i] := ls[i]$
		\end{itemize}
	\item []
	return $L$
	\end{itemize}
\end{itemize}

The transition function is therefore as follows:

\begin{itemize}
\item []
$transition-function(ls, stack):$
	\begin{itemize}
	\item []
	$start, end := stack[0]$
	\item []
	$pivot_index := n, such that 1 \leq n \leq length(ls)$
	\item[]
	$ls' := Partition (ls, start, end, ls[pivot_index])$
	\item []
	$ if (pivot - 1) > start:$
		\begin{itemize}
		\item []
		$stack' = [(start, pivot - 1)] ++ stack$
		\end{itemize}
	\item []
	$ if (pivot + 1) < end:$
		\begin{itemize}
		\item []
		$stack' = [(pivot + 1, end)] ++ stack$
		\end{itemize}
	\item []
	return $(ls', stack')$
	\end{itemize}
\end{itemize}

\section{Proof of Termination}

Since the transition system terminates when the list $stack$ is empty, proof of termination equates to a proof that stack will eventually become empty.

At each iteration

As element 2 is a decreasing sequence

As element 1 is an increasing sequence

\section{Proof of Correctness}

Correctness rests on the invariant that at any state of the system, a pivot element can be chosen from list $ls$ such that:

\begin{itemize}
\item
Elements before the pivot's index are lesser than pivot
\item
Elements after the pivot's index are greater than pivot.
\end{itemize}

\subsection{Base Case}

When $start = end$, i.e: the list is a singlet list, then invariant holds.

\subsection{Inductive Step}

Assume $quicksort(ls)$ is correct for a list of length $N$.

Take $quicksort(ls')$, which is a list of length $N + 1$.

Choosing a pivot from $ls'$ and then calling partition on $ls'$ gives us two sub-lists, one $greater$ than pivot, the other $lesser$ than pivot.

$0 \leq length(greater), length(lesser) \leq N$. Therefore, by assumption, $quicksort(greater)$ and $quicksort(lesser)$ are correct.

$ls' = lesser ++ [pivot] ++ greater$, where all elements in $greater$ are greater than pivot, and all elements in $lesser$ are lesser than pivot. Therefore, $quicksort(ls')$ is correct if $quicksort(ls)$ is correct.

Thus proven inductively, the invariant holds for all lists of length greater than or equal to 1.

Therefore $quicksort$ system is correct.


% --------------------------------------------------------------
%     You don't have to mess with anything below this line.
% --------------------------------------------------------------
 
\end{document}
